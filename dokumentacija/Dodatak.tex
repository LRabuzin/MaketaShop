\chapter*{Dodatak: Prikaz aktivnosti grupe}
		\addcontentsline{toc}{chapter}{Dodatak: Prikaz aktivnosti grupe}
		
		\section*{Dnevnik sastajanja}
		
		\textbf{\textit{Kontinuirano osvježavanje}}\\
		
		 \textit{U ovom dijelu potrebno je redovito osvježavati dnevnik sastajanja prema predlošku.}
		
		\begin{packed_enum}
			\item  sastanak
			
			\item[] \begin{packed_item}
				\item Datum: 5. listopada 2020.
				\item Prisustvovali: T.Hrestak, L.Novački, P.Pažur, L.Rabuzin, S.Sičić, I.Šarić, F.Zmiša
				\item Teme sastanka:
				\begin{packed_item}
					\item  uvodni sastanak s asistentom
					\item  opis i detaljnije objašnjenje projektnog zadatka
					\item  osnovne informacije o organizaciji projekta
					\item  razjašnjene temeljne dileme oko funkcionalnosti
				\end{packed_item}
			\end{packed_item}
		
			\item  sastanak
			
			\item[] \begin{packed_item}
				\item Datum: 14. listopada 2020.
				\item Prisustvovali: T.Hrestak, L.Novački, P.Pažur, L.Rabuzin, S.Sičić, I.Šarić, F.Zmiša
				\item Teme sastanka:
				\begin{packed_item}
					\item  pojašnjavanje rada s gitom i \LaTeX om
					\item  postavljena pravila o radu s dokumentacijom
					\item  uspostavljeni kanali komunikacije i organizacije (Discord i Trello)
					\item  podijeljen posao oko rada na poglavlju 2 i 3
				\end{packed_item}
			\end{packed_item}
		
			\item  sastanak
			
			\item[] \begin{packed_item}
				\item Datum: 17. listopada 2020.
				\item Prisustvovali: T.Hrestak, L.Novački, P.Pažur, L.Rabuzin, S.Sičić, I.Šarić, F.Zmiša
				\item Teme sastanka:
				\begin{packed_item}
					\item  zajednički rad na definiciji obrazaca uporabe
				\end{packed_item}
			\end{packed_item}
		
			\item  sastanak
			
			\item[] \begin{packed_item}
				\item Datum: 19. listopada 2020.
				\item Prisustvovali: T.Hrestak, L.Novački, P.Pažur, L.Rabuzin, S.Sičić, I.Šarić, F.Zmiša
				\item Teme sastanka:
				\begin{packed_item}
					\item  planiranje daljnih aktivnosti
					\item  delegacija posla:
					\begin{packed_item}
						\item  UML dijagrami za obrasce uporabe - Hrestak i Zmiša
						\item  prebacivanje opisa obrazaca uporabe iz google docs-a u službenu dokumentaciju: Sičić
						\item  sekvencijski dijagrami - Novački
						\item  privremeno sučelje - Rabuzin, Sičić i Šarić
						\item  baza podataka - Pažur
					\end{packed_item}
				\end{packed_item}
			\end{packed_item}
			
			%
			
		\end{packed_enum}
		
		\eject
		\section*{Tablica aktivnosti}
		
			\textbf{\textit{Kontinuirano osvježavanje}}\\
			
			 \textit{Napomena: Doprinose u aktivnostima treba navesti u satima po članovima grupe po aktivnosti.}
					
						
			
			\begin{longtabu} to \textwidth {|X[7, l]|X[1, c]|X[1, c]|X[1, c]|X[1, c]|X[1, c]|X[1, c]|X[1, c]|}
								
				\cline{2-8} \multicolumn{1}{c|}{\textbf{}} &     \multicolumn{1}{c|}{\rotatebox{90}{\textbf{Lovro Rabuzin }}} & \multicolumn{1}{c|}{\rotatebox{90}{\textbf{Tvrtko Hrestak }}} &	\multicolumn{1}{c|}{\rotatebox{90}{\textbf{Leon Novački }}} &	\multicolumn{1}{c|}{\rotatebox{90}{\textbf{Patrik Pažur }}} &
				\multicolumn{1}{c|}{\rotatebox{90}{\textbf{Sara Sičić }}} &
				\multicolumn{1}{c|}{\rotatebox{90}{\textbf{Ivona Šarić }}} &	\multicolumn{1}{c|}{\rotatebox{90}{\textbf{Filip Zmiša }}} \\ \hline 
				\endfirsthead
				
			
				\cline{2-8} \multicolumn{1}{c|}{\textbf{}} &     \multicolumn{1}{c|}{\rotatebox{90}{\textbf{Ime Prezime voditelja}}} & \multicolumn{1}{c|}{\rotatebox{90}{\textbf{Ime Prezime }}} &	\multicolumn{1}{c|}{\rotatebox{90}{\textbf{Ime Prezime }}} &
				\multicolumn{1}{c|}{\rotatebox{90}{\textbf{Ime Prezime }}} &	\multicolumn{1}{c|}{\rotatebox{90}{\textbf{Ime Prezime }}} &
				\multicolumn{1}{c|}{\rotatebox{90}{\textbf{Ime Prezime }}} &	\multicolumn{1}{c|}{\rotatebox{90}{\textbf{Ime Prezime }}} \\ \hline 
				\endhead
				
				
				\endfoot
							
				 
				\endlastfoot
				
				Upravljanje projektom 		&  &  &  &  &  &  & \\ \hline
				Opis projektnog zadatka 	& 4.5 &  &  &  &  & 4.5 & \\ \hline
				
				Funkcionalni zahtjevi       & 1 &  &  &  & 1 &  &  \\ \hline
				Opis pojedinih obrazaca 	& 2 & 2 & 2 & 2 & 2 & 2 & 2 \\ \hline
				Dijagram obrazaca 			&  & 3.5 &  &  &  & & 3.5  \\ \hline
				Sekvencijski dijagrami 		&  &  &  &  &  &  &  \\ \hline
				Opis ostalih zahtjeva 		&  &  &  &  &  &  &  \\ \hline

				Arhitektura i dizajn sustava	 &  &  &  &  &  &  &  \\ \hline
				Baza podataka				&  &  &  &  &  &  &   \\ \hline
				Dijagram razreda 			&  &  &  &  &  &  &   \\ \hline
				Dijagram stanja				&  &  &  &  &  &  &  \\ \hline
				Dijagram aktivnosti 		&  &  &  &  &  &  &  \\ \hline
				Dijagram komponenti			&  &  &  &  &  &  &  \\ \hline
				Korištene tehnologije i alati 		&  &  &  &  &  &  &  \\ \hline
				Ispitivanje programskog rješenja 	&  &  &  &  &  &  &  \\ \hline
				Dijagram razmještaja			&  &  &  &  &  &  &  \\ \hline
				Upute za puštanje u pogon 		&  &  &  &  &  &  &  \\ \hline 
				Dnevnik sastajanja 			&  &  &  &  &  &  &  \\ \hline
				Zaključak i budući rad 		&  &  &  &  &  &  &  \\  \hline
				Popis literature 			&  &  &  &  &  &  &  \\  \hline
				&  &  &  &  &  &  &  \\ \hline \hline
				\textit{Dodatne stavke kako ste podijelili izradu aplikacije} 			&  &  &  &  &  &  &  \\ \hline
				\textit{npr. izrada početne stranice} 				&  &  &  &  &  &  &  \\ \hline 
				\textit{izrada baze podataka} 		 			&  &  &  &  &  &  & \\ \hline 
				\textit{spajanje s bazom podataka} 							&  &  &  &  &  &  &  \\ \hline
				\textit{back end} 							&  &  &  &  &  &  &  \\  \hline
				 							&  &  &  &  &  &  &\\  \hline
				
				
			\end{longtabu}
					
					
		\eject
		\section*{Dijagrami pregleda promjena}
		
		\textbf{\textit{dio 2. revizije}}\\
		
		\textit{Prenijeti dijagram pregleda promjena nad datotekama projekta. Potrebno je na kraju projekta generirane grafove s gitlaba prenijeti u ovo poglavlje dokumentacije. Dijagrami za vlastiti projekt se mogu preuzeti s gitlab.com stranice, u izborniku Repository, pritiskom na stavku Contributors.}
		
	