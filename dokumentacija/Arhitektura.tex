\chapter{Arhitektura i dizajn sustava}
		
		%\textbf{\textit{dio 1. revizije}}\\

		%\textit{ Potrebno je opisati stil arhitekture te identificirati: podsustave, preslikavanje na radnu platformu, spremišta podataka, mrežne protokole, globalni upravljački tok i sklopovsko-programske zahtjeve. Po točkama razraditi i popratiti odgovarajućim skicama:}
	%\begin{itemize}
	%	\item 	\textit{izbor arhitekture temeljem principa oblikovanja pokazanih na predavanjima (objasniti zašto ste baš odabrali takvu arhitekturu)}
		%\item 	\textit{organizaciju sustava s najviše razine apstrakcije (npr. klijent-poslužitelj, baza podataka, datotečni sustav, grafičko sučelje)}
	%	\item 	\textit{organizaciju aplikacije (npr. slojevi frontend i backend, MVC arhitektura) }		
%	\end{itemize}

		Arhitektura našeg sustava dijeli se na tri podsustava: baza podataka, web aplikacija i web poslužitelj.
			
		\underbar{Web preglednik} program je koji služi kao posrednik između korisnika i web poslužitelja. Korisniku omogućuje pregled web-stranica te multimedijskih sadržaja vezanih uz njih. Svaka stranica pisana je u nekom kodu koji prosječnom korisniku ništa ne znači, no kako je svaki internetski preglednik ujedno i prevoditelj, on prikazuje stranicu u obliku koja je svakome razumljiva. Na taj način korisnik šalje zahtjeve web poslužitelju.
		
		\underbar{Web poslužitelj} kao osnovnu zadaću ima ostvarivanje komunikacije između klijenta i aplikacije.  Ta komunikacija ostvarena je HTTP-om (engl. HyperText Transfer Protocol). Upravo je web poslužitelj temelj rada web aplikacije. On ju pokreće i prosljeđuje zahtjeve zaprimljene od web preglednika.
		
		\underbar{Web aplikacija} koju korisnik koristi obrađuje njegove zahtjeve. Ukoliko je potrebno za obradu zahtjeva, web aplikacija komunicira s poslužiteljem baze podataka koji joj dohvaća i prosljeđuje potrebne podatke. Potom web aplikacija vraća odgovor u obliku HTML dokumenta te web preglednik to prikazuje korisniku u odgovarajućem formatu.
		
		Za izradu naše web aplikacije odlučili smo se za programski jezik Python s njegovim radnim okvirom Django. Koristili smo Bootstrap, HTML, CSS i JavaScript za prikaz web-stranica. Baza podataka implementirana je kroz PostgreSQL. Arhitektura sustava temeljit će se na MVT (eng. \textit{Model View Template}) obrascu koji se tek marginalno razlikuje od MVC (eng. \textit{Model View Controller}) obrasca. S obzirom da je MVT koncept podržan od strane Djanga, na raspolaganju su nam gotovi predlošci te nam znatno olakšavaju razvoj web aplikacije.
	
		Zahvaljujući nezavisnosti razvoja pojedinih djelova aplikacije možemo jednostavnije ispitivati i razvijati sustav, kao i dodavati nova svojstva. Kao što se može pretpostaviti, MVT koncept sastoji se od triju komponenti. "Model" i "View" na strani su poslužitelja i nisu vidljivi korisniku, dok je "Template" vidljiv na korisničkoj strani. "Model" je središnja jkomponenta sustava te pristupa bazi podataka. Pravilno formatira podatke dobivene od strane "View"-a te ih prosljeđuje bazi podataka i obrnuto. "View" prima podatke i zahtjeve kao što su "POST" i "GET" s klijentske strane. Također pravilno formatira primljene podatke te komunicira s druge dvije komponente MVT koncepta. "Template" služi za prikazivanje sadržaja na web-stranici. Sadrži statičke i dinamičke definicije prikaza sadržaja.
		
		\begin{figure} [!h] %ovaj [!h] je potreban za fiksiranje slike točno na ovo mjesto u pdf-u.
			\centering
			\includegraphics[width=0.7\linewidth]{slike/MVT}
			\caption{MVT koncept}
			\label{fig:mvt}
		\end{figure}
		
		\eject		

				
		\section{Baza podataka}
			
		Koristimo relacijsku bazu podataka napisanu u jeziku SQL te je ostvarujemo kroz postgreSQL. Pregled cijele baze imamo preko programa pgAdmin 4. Tamo gledamo spremaju li se promjene, jesu li one smislene te dodajemo specifičan sadržaj za testiranje. Django preko komponente "Model" pristupa bazi te je ona strukturno cijela sadržana u datoteci models.py.
		
			\subsection{Opis tablica}
			
				
				\begin{longtabu} to \textwidth {|X[10, l]|X[6, l]|X[20, l]|}
					
					\hline \multicolumn{3}{|c|}{\textbf{Korisnik}}	 \\[3pt] \hline
					\endfirsthead
					
					\hline \multicolumn{3}{|c|}{\textbf{Korisnik}}	 \\[3pt] \hline
					\endhead
					
					\hline 
					\endlastfoot
					
					\cellcolor{LightGreen}korisnikid & INT	&  	Redni broj korisnika (primarni ključ). 	\\ \hline
					email	& VARCHAR &  Korisnikov e-mail  (maksimalno 100 znakova). \\ \hline
					korisnickoime & VARCHAR & Ime koje predstavlja korisnika na stranici (maksimalno 20 znakova). \\ \hline 
					lozinka & VARCHAR & Korisnikova lozinka za prijavu (maksimalno 20 znakova).  \\ \hline 
					jeadmin & BOOL & Ima li korisnik ovlasti administratora? \\ \hline 
					adresa & VARCHAR & Korisnikova adresa (maksimalno 100 znakova). \\ \hline  
					datumregistracije & DATE & Datum korisnikove registracije.  \\ \hline 
					adresaprivatna & BOOL & Želi li korisnik javno prikazati svoju adresu na svome profilu? \\ \hline 
					datumregistracije
					privatan & BOOL & Želi li korisnik javno prikazati svoj datum registracije na svome profilu?  \\ \hline 
					slikaprivatna & BOOL & Želi li korisnik javno prikazati svoju slie ime i prezime?  \\ \hline 
					emailprivatan & BOOL & Želi li korisnik javno prikazati svoj e-mail na svome profilu?   \\ \hline 
					imeprezimeprivatno & BOOL & Želi li korisnik javno prikazati svoje ime i prezime na svome profilu?  \\ \hline 	 
					ime & VARCHAR &  Korisnikovo ime (maksimalno 50 znakova).  \\ \hline 
					prezime  & VARCHAR & Korisnikovo prezime (maksimalno 50 znakova).  \\ \hline 
					dozvoljenpristup & BOOL & Ima li korisnik dozvoljen pristup stranici?  \\ \hline 
					kkpaypal & BOOL & Plaća li korisnik paypalom?  \\ \hline 
					kkbroj & VARCHAR & Broj kreditne kartice korisnika. (16 znakova)  \\ \hline
					kkimeprezime & VARCHAR & Ime i prezime na kreditnoj kartici korisnika. (maksimalno 100 znakova)  \\ \hline
					kkistek & VARCHAR & Mjesec i godina isteka kreditne kartice. (5 znakova)  \\ \hline
					\cellcolor{LightBlue} 
					\cellcolor{LightBlue} 
					profilnaid & INT & Redni broj medijske datoteke koja sadrži korisnikovu profilnu fotografiju. (strani ključ)  \\ \hline 
									
					
				\end{longtabu}
			
				\begin{longtabu} to \textwidth {|X[10, l]|X[6, l]|X[20, l]|}
				
				\hline \multicolumn{3}{|c|}{\textbf{Komentar}}	 \\[3pt] \hline
				\endfirsthead
				
				\hline \multicolumn{3}{|c|}{\textbf{Komentar}}	 \\[3pt] \hline
				\endhead
				
				\hline 
				\endlastfoot
				
				\cellcolor{LightGreen}komentarid & INT	&  	Redni broj komentara (primarni ključ). 	\\ \hline
				sadrzaj	& VARCHAR &  Tekstualni sadržaj komentara (maksimalno 300 znakova). \\ \hline
				\cellcolor{LightGreen}
				korisnikid & INT & Redni broj korisnika koji je ostavio komentar. (strani ključ) \\ \hline 
				\cellcolor{LightBlue}
				pricaid & INT & Redni broj priče na kojoj je ostavljen komentar. (strani ključ) \\ \hline 	
				
				\end{longtabu}
			
				\begin{longtabu} to \textwidth {|X[10, l]|X[6, l]|X[20, l]|}
					
				\hline \multicolumn{3}{|c|}{\textbf{KorisnikDislajkaopricu}}	 \\[3pt] \hline
				\endfirsthead
				
				\hline \multicolumn{3}{|c|}{\textbf{KorisnikDislajkaopricu}}	 \\[3pt] \hline
				\endhead
				
				\hline 
				\endlastfoot
				\cellcolor{LightGreen}id & INT	&  	Redni broj unosa. (primarni ključ)	\\ \hline
				\cellcolor{LightBlue}
				korisnikid & INT & Redni broj korisnika koji je označio sa "ne sviđa mi se". (strani ključ) \\ \hline 
				\cellcolor{LightBlue}
				pricaid & INT & Redni broj priče koja je označena sa "ne sviđa mi se". (strani ključ) \\ \hline 	
					
				\end{longtabu}
			
				\begin{longtabu} to \textwidth {|X[10, l]|X[6, l]|X[20, l]|}
					
					\hline \multicolumn{3}{|c|}{\textbf{KorisnikLajkaopricu}}	 \\[3pt] \hline
					\endfirsthead
					
					\hline \multicolumn{3}{|c|}{\textbf{KorisnikLajkaopricu}}	 \\[3pt] \hline
					\endhead
					
					\hline 
					\endlastfoot
					
					\cellcolor{LightGreen}id & INT	&  	Redni broj unosa. (primarni ključ)	\\ \hline
					\cellcolor{LightBlue}
					korisnikid & INT & Redni broj korisnika koji je označio priču sa "sviđa mi se". (strani ključ) \\ \hline 
					\cellcolor{LightBlue}
					pricaid & INT & Redni broj priče koja je označena sa "sviđa mi se". (strani ključ) \\ \hline 	
					
				\end{longtabu}
			
			\begin{longtabu} to \textwidth {|X[10, l]|X[6, l]|X[20, l]|}
				
				\hline \multicolumn{3}{|c|}{\textbf{Maketa}}	 \\[3pt] \hline
				\endfirsthead
				
				\hline \multicolumn{3}{|c|}{\textbf{Maketa}}	 \\[3pt] \hline
				\endhead
				
				\hline 
				\endlastfoot
				
				\cellcolor{LightGreen}maketaid & INT	&  	Redni broj makete (primarni ključ). 	\\ \hline
				ime & VARCHAR & Ime makete. (100 znakova) \\ \hline 
				dimenzije & VARCHAR & Dimenzije makete u centimetrima. \\ \hline 
				opis & VARCHAR & Opis makete. (160 znakova) \\ \hline 
				\cellcolor{LightBlue}
				mediaid & INT & Redni broj medijske datoteke koja sadrži sliku makete. (strani ključ) \\ \hline 	
				\cellcolor{LightBlue}
				vrsta & INT & Vrsta makete. (strani ključ) \\ \hline 
				prihvacena & BOOL & Je li maketa prihvaćena?  \\ \hline 
					
				
			\end{longtabu}
		
			\begin{longtabu} to \textwidth {|X[10, l]|X[6, l]|X[20, l]|}
				
				\hline \multicolumn{3}{|c|}{\textbf{MaketaKupljena}}	 \\[3pt] \hline
				\endfirsthead
				
				\hline \multicolumn{3}{|c|}{\textbf{MaketaKupljena}}	 \\[3pt] \hline
				\endhead
				
				\hline 
				\endlastfoot
				
				\cellcolor{LightGreen}id & INT	&  	Redni broj kupovine makete (primarni ključ). 	\\      \hline			
				kolicina & INT & Broj istih maketa kupljenih pri ovoj narudžbi. \\ \hline 
				\cellcolor{LightBlue}
				maketaid & INT & Redni broj navedene makete. (strani ključ) \\ \hline 
				\cellcolor{LightBlue}
				materijalid & INT & Redni broj materijala od kojeg je izgrađena navedena maketa. (strani ključ) \\ \hline 	
				\cellcolor{LightBlue}
				transakcijaid & INT & Redni broj transakcije u pitanju. (strani ključ) \\ \hline 
				
			\end{longtabu}
		
			\begin{longtabu} to \textwidth {|X[10, l]|X[6, l]|X[20, l]|}
				
				\hline \multicolumn{3}{|c|}{\textbf{Materijal}}	 \\[3pt] \hline
				\endfirsthead
				
				\hline \multicolumn{3}{|c|}{\textbf{Materijal}}	 \\[3pt] \hline
				\endhead
				
				\hline 
				\endlastfoot
				
				\cellcolor{LightGreen}materijalid & INT	&  	Redni broj materijala (primarni ključ). 	\\      \hline			
				ime & VARCHAR & Ime materijala (maksimalno 100 znakova). \\ \hline 
				
			\end{longtabu}
		
			\begin{longtabu} to \textwidth {|X[10, l]|X[6, l]|X[20, l]|}
			
			\hline \multicolumn{3}{|c|}{\textbf{Media}}	 \\[3pt] \hline
			\endfirsthead
			
			\hline \multicolumn{3}{|c|}{\textbf{Media}}	 \\[3pt] \hline
			\endhead
			
			\hline 
			\endlastfoot
			
			\cellcolor{LightGreen}mediaid & INT	&  	Redni broj medijske datoteke (primarni ključ). 	\\      \hline		
			vrstamedije & VARCHAR & Tip datoteke. (slika, tekst, video) \\ \hline 	
			putdodatoteke & VARCHAR & Relativni put do datoteke u repozitoriju. \\ \hline 
			
			\end{longtabu}
				
				\begin{longtabu} to \textwidth {|X[10, l]|X[6, l]|X[20, l]|}
				
				\hline \multicolumn{3}{|c|}{\textbf{MultimedijaPriče}}	 \\[3pt] \hline
				\endfirsthead
				
				\hline \multicolumn{3}{|c|}{\textbf{MultimedijaPriče}}	 \\[3pt] \hline
				\endhead
				
				\hline 
				\endlastfoot
				\cellcolor{LightGreen}id & INT	&  	Redni broj multimedije u priči. (primarni ključ). 	\\      \hline			
				poredakuprici & INT & Broj u poretku po kojem se slaže multimedija u nekoj priči. \\ \hline 
				\cellcolor{LightBlue}
				mediaid & INT & Redni broj medijske datoteke u pitanju. (strani ključ) \\ \hline 
				\cellcolor{LightBlue}
				pricaid & INT & Redni broj priče u pitanju. (strani ključ) \\ \hline 
				
			\end{longtabu}
		
		\begin{longtabu} to \textwidth {|X[10, l]|X[6, l]|X[20, l]|}
			
			\hline \multicolumn{3}{|c|}{\textbf{NapravljenaOd}}	 \\[3pt] \hline
			\endfirsthead
			
			\hline \multicolumn{3}{|c|}{\textbf{NapravljenaOd}}	 \\[3pt] \hline
			\endhead
			
			\hline 
			\endlastfoot
			\cellcolor{LightGreen}id & INT	&  	Redni broj unosa. (primarni ključ). 	\\      \hline			
			cijena & FLOAT & Cijena makete napravljene od specifičnog materijala. \\ \hline 
			\cellcolor{LightBlue}
			maketaid & INT & Redni broj makete u pitanju. (strani ključ) \\ \hline 
			\cellcolor{LightBlue}
			materijalid & INT & Redni broj materijala u pitanju. (strani ključ) \\ \hline 
			brojuskladistu & INT & Broj dostupnih maketa u skladištu. \\ \hline 
			
		\end{longtabu}
	
			\begin{longtabu} to \textwidth {|X[10, l]|X[6, l]|X[20, l]|}
				
				\hline \multicolumn{3}{|c|}{\textbf{Priča}}	 \\[3pt] \hline
				\endfirsthead
				
				\hline \multicolumn{3}{|c|}{\textbf{Priča}}	 \\[3pt] \hline
				\endhead
				
				\hline 
				\endlastfoot
				
				\cellcolor{LightGreen}pricaid & INT	&  	Redni broj priče (primarni ključ). 	\\      \hline			
				naslovprice & VARCHAR & Naslov priče (maksimalno 100 znakova). \\ \hline 
				datumprice & DATE & Datum objave priče. \\ \hline
				objavljena & BOOL & Je li priča objavljena? \\ \hline 
				\cellcolor{LightBlue}
				maketaid & INT & Redni broj makete u pitanju. (strani ključ) \\ \hline 
				\cellcolor{LightBlue}
				autorid & INT & Redni broj autora priču. (strani ključ) \\ \hline 
				\cellcolor{LightBlue}
				predloziopricuid & INT & Redni broj osobe koja je predložila priču. (strani ključ)
				
			\end{longtabu}
		
			\begin{longtabu} to \textwidth {|X[10, l]|X[6, l]|X[20, l]|}
			
			\hline \multicolumn{3}{|c|}{\textbf{Tema}}	 \\[3pt] \hline
			\endfirsthead
			
			\hline \multicolumn{3}{|c|}{\textbf{Tema}}	 \\[3pt] \hline
			\endhead
			
			\hline 
			\endlastfoot
			
			\cellcolor{LightGreen}temaid & INT	&  	Redni broj teme (primarni ključ). 	\\      \hline			
			ime & VARCHAR & Ime teme (maksimalno 100 znakova). \\ \hline 		
			\cellcolor{LightBlue}
			teksttemeid & INT & Redni broj medijske datoteke koja sadrži tekst teme. (strani ključ) \\ \hline 
			odobrena & BOOL & Je li priča objavljena? \\ \hline 
			
		\end{longtabu}
	
		\begin{longtabu} to \textwidth {|X[10, l]|X[6, l]|X[20, l]|}
			
			\hline \multicolumn{3}{|c|}{\textbf{Transakcija}}	 \\[3pt] \hline
			\endfirsthead
			
			\hline \multicolumn{3}{|c|}{\textbf{Transakcija}}	 \\[3pt] \hline
			\endhead
			
			\hline 
			\endlastfoot
			
			\cellcolor{LightGreen}transakcijaid & INT	&  	Redni broj transakcije(primarni ključ). 	\\      \hline			
			ime & VARCHAR & Ime kupca (maksimalno 100 znakova). \\ \hline 
			prezime & VARCHAR & Prezime kupca (maksimalno 100 znakova). \\ \hline 
			adresa & VARCHAR & Adresa kupca (maksimalno 100 znakova). \\ \hline 	
			brojracuna & VARCHAR & Broj računa kupca ( 21 znak). \\ \hline 	
			ukupaniznos & FLOAT & Ukupan iznos transakcije \\ \hline 
			\cellcolor{LightBlue}
			korisnik & INT & Redni broj korisnika koji je pokrenuo transakciju. (strani ključ) \\ \hline 
			datumizvedena & DATE & Datum transakcije.  \\ \hline 
			
		\end{longtabu}
	
		\begin{longtabu} to \textwidth {|X[10, l]|X[6, l]|X[20, l]|}
			
			\hline \multicolumn{3}{|c|}{\textbf{VrstaMakete}}	 \\[3pt] \hline
			\endfirsthead
			
			\hline \multicolumn{3}{|c|}{\textbf{VrstaMakete}}	 \\[3pt] \hline
			\endhead
			
			\hline 
			\endlastfoot
			\cellcolor{LightGreen}id & INT	&  	Redni broj vrste makete. (primarni ključ). 	\\      \hline			
			ime & VARCHAR & Ime vrste makete. (maksimalno 20 znakova). \\ \hline 	
			
		\end{longtabu}
		
			
			
			
			\subsection{Dijagram baze podataka}
			\vspace{\baselineskip}
			\vspace{\baselineskip}
			\vspace{\baselineskip}
			\vspace{\baselineskip}
			\vspace{\baselineskip}
			\vspace{\baselineskip}
			\vspace{\baselineskip}
			\vspace{\baselineskip}
			\vspace{\baselineskip}
			\vspace{\baselineskip}
				\includegraphics[width=1\linewidth]{slike/bazaRelShem}
			
			\eject
			
			
		\section{Dijagram razreda}
		
		Slike 4.2, 4.3 i 4.4 predstavljaju implementaciju razreda korištenih u backendu za prijenos podataka u korištenoj MTV arhitekturi. Slika 4.2 sadrži sve korištene razrede koji predstavljaju View tipove. Na slici 4.3 prikazane su Data Transfer Object tipovi podataka koji primaju podatake iz modela.  Slika 4.4 prikazuje razred modela onako kako su zapamćeni u bazi podataka te služe za direktnu interakciju s njom. WeTriedContext predstavlja sveukupan sadržaj korištene baze podataka.

		\begin{figure}[H]
			\centering
			\includegraphics[width=1.1\linewidth, height=0.4\textheight]{slike/dijagram_razreda_1_1}
			\caption{Dijagram razreda - Views - Prvi}
			\label{fig:dijagramrazreda1}
		\end{figure}
		
		\begin{figure}[H]
			\centering
			\includegraphics[width=1.1\linewidth, height=0.4\textheight]{slike/dijagram_razreda_1_2}
			\caption{Dijagram razreda - Views - Drugi}
			\label{fig:dijagramrazreda1}
		\end{figure}
		
		\begin{figure}[H]
			\centering
			\includegraphics[width=1.0\linewidth]{slike/dijagram_razreda_2}
			\caption{Dijagram razreda - DTO}
			\label{fig:dijagramrazreda2}
		\end{figure}
	
		\begin{figure}[H]
			\centering
			\includegraphics[width=1.0\linewidth]{slike/dijagram_razreda_3}
			\caption{Dijagram razreda - Models}
			\label{fig:dijagramrazreda3}
		\end{figure}
	
		\eject

	\section{Dijagram stanja}
	
		Prikazan je dijagram stanja za registriranog korisnika. Zaglavlje stranice je uvijek dostupno i kroz nju korisnik može uvijek otići na "Početnu stranicu", "Moj Profil", "Webshop", "Moje transakcije", "Košarica" i "Sandučić. Također, kroz zaglavlje stranice registrirani korisnik se uvijek može i odjaviti. 
		
		Nakon prijave, klijentu se prikazuje početna stranica na kojoj može pregledati priče. Pritiskom na pojedinu priču korisnik može pročitati cijelu priču i označiti da mu se priča sviđa ili ne sviđa, i može ostaviti komentar na priču. 
		
		Na stranici "Webshop", korisnik ima pregled na makete za prodaju i gumbe "Kupi" s kojima dodaje makete u košaricu. Klikom na maketu korisnik dobije detaljniji pregled pojedine makete gdje mu je također ponuđen gumb "Kupi". Klikom na "Kupi" korisniku se prikazuje njegova košarica. Ako korisnik želi nastaviti kupovinu, "Nastavi kupovinu" će ga odvesti nazad na Webshop, a "Dovrši kupnju" će ga odvesti na stranicu za plaćanje. Na stranici za plaćanje korisnik unosi podatke o plaćanju i završava kupnju. 
		
		Klikom na "Moj Profil" prikazuju mu se njegovi podatci i potvrdni okviri s kojima mijenja postavke privatnosti. Klikom na "Spremi postavke privatnosti", korisnik sprema postavke. Klikom na "browse" korisnik može snimiti sliku za svog računala i sa klikom na "Spremi" postaviti tu sliku kao profilnu sliku računa. 
		
		Klikom na "Sandučić" korisniku se nude poveznice na stranice gdje može predlagati temu, priču ili naručiti maketu po narudžbi. Također, može vidjeti povijest svojih predlaganih tema, priča i dogovora s adminom o naručenim maketama.
		
		
		\begin{figure}[H]
			\centering
			\includegraphics[width=1\linewidth]{slike/Dijagram_stanja_registrirani_klijent}
			\caption{Dijagram stanja - registrirani klijent}
			\label{fig:dijagramstanja}
		\end{figure}
		\eject
		\pagebreak
		
	\section{Dijagram aktivnosti}
		
		 Na dijagramu aktivnosti 4.7 prikazan je proces pregleda priče. Korisnik se prijavi u sustav te na početnoj stranici odabire priču makete koju želi detaljnije pregledati. U detaljnijem pregledu priče osim opisa makete korisniku se prikazuje broj lajkova i dislajkova za priču te komentari. Sam korisnik može odlučiti sviđa li mu se priča ili ne te ostaviti svoj komentar. Pregled priče prestaje odlaskom na neku drugu stranicu.
		
		
		\begin{figure}[!h]
			\centering
			\includegraphics[width=1\linewidth]{slike/dijagram_aktivnosti}
			\caption{Dijagram aktivnosti - Pregled priče}
			\label{fig:dijagramaktivnosti}
		\end{figure}
		\eject
	
	\section{Dijagram komponenti}
		
		Na dijagramu komponenti na slici 4.8 prikazana je organizacija i međuovisnost komponenti te interne strukture i odnosi komponenti prema okolini. Preko DJANGA poslužuju se datoteke koje pripadaju frontend djelu aplikacije. Podaci iz tablica iz baze podataka se preko QuerySet APIja šalju se u DTO obliku dalje u aplikaciju. 
		
		
		\begin{figure}[!h]
			\centering
			\includegraphics[width=1\linewidth]{slike/Dijagram_komponenti}
			\caption{Dijagram komponenti}
			\label{fig:dijagramaktivnosti}
		\end{figure}
		\eject