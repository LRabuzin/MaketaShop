\chapter{Zaključak i budući rad}
		
		Zadatak dodijeljen našoj grupi bio je razvoj web aplikacije za objavljivanje multimedijskog sadržaja o maketama te mogućnost prodavanja maketa. Izrada samog projekta trajala je 15 tjedana kroz kojih je većinom prevladavao rad na implementaciji programskog rješenja, a dio vremena odvojili smo na pisanje dokumentacije projekta. Na samom početku projekta, izdvojili smo dio vremena za pisanje projektne dokumentacije koja je služila kao dobar predložak u procesu pisanja programske potpore. Projekt nam se ugrubo sastojao od dvije faze.
		
		\textbf	Prva faza većinom se sastojala od upoznavanja s novim pojmovima i tehnologijima. Nakon odabira željenih tehnologija, krenuli smo s implementacijom jednostavnijih funkcionalnosti naše web aplikacije kao što su jednostavnije korisničko sučelje, registracija i odjavljivanje korisnika, pregled vlastitog profila, početna stranica i online trgovina s pregledom priča o maketama. Početna stranica i online trgovina sadržavale su statičke podatke te su služile kao svojevrsan prototip za daljnji razvoj stranice. Uz dokumentaciju, veliki dio početne faze uključivalo je detaljno oblikovanje baze podataka.
		
		\textbf Druga faza započela je prebacivanjem backenda na objektno orijentiranu paradigmu. Nadalje, bilo je potrebno dinamički povezati sadržaje na stranicama. Osim nekoliko izmjena u bazi podataka započeli smo s implementacijom zahtjevnijih funkcionalnosti web aplikacije. Jedan od glavnih zadataka bio je napraviti sandučić koji služi za interakciju administratora i registriranog korisnika. Od zahtjevnijih zadataka ističe se implementacija procesa kupovine koji uključuje pregled i odabir maketa, prikazivanje željenih maketa u košarici te ispunjavanje traženog formulara koji predstavlja plaćanje putem interneta. Stekli smo znanje o postavljanju multimedijskog sadržaja na stranicu i daljnjem rukovanju tog sadržaja.
		
		\textbf Sudjelovanje na ovom projektu naučilo nas je komunikaciji između backenda i frontenda. Dobra podjela rada i organizacija bili su od velikog značaja za uspješno izvršavanje dodijeljenih zadatka. Glavna komunikacija vršila se putem platforme Discord. Moguća proširenja uključivala bi dodatne interaktivne funkcionalnosti korisnika i administratora. Tijekom projekta stekli smo bitna znanja o razvoju web aplikacija. Mislimo da uvijek postoji prostora za usavršavanje aplikacije, no zadovoljni smo postignutim rezultatima.
		
		
		\eject 