\chapter{Opis projektnog zadatka}

		Cilj ovog projekta stvoriti je korisničku aplikaciju koja služi za objavljivanje multimedijskog sadržaja o izrađenim maketama te njihovu prodaju. Aplikacija se uglavnom sastoji od funkcionalnosti koje služe za objavu sadržaja na stranici, pohranu podataka o korisnicima te prodaju artikala.
		
		Osnovna je funkcionalnost aplikacije objavljivanje priča o izrađenim maketama. Korisnik priču može objaviti kao multimedijski sadržaj. Ponuđeno mu je birati između slika, videa ili teksta ili pak nekom kombinacijom navedenih formata. Sustav omogućuje administratorima objavljivanje sadržaja dok ih registrirani korisnici mogu samo predlagati. Predložene priče administratori pregledavaju te im se nudi opcija odobravanja objave priče. Već spomenuti registrirani korisnici, kao i oni neregistrirani, mogu komentirati objavljene sadržaje. Registriranim korisnicima se nudi opcija i označavanja priče ocjenama “sviđa mi se” ili “ne sviđa mi se”. Korisnici mogu samo vidjeti broj ocjena za pojedinu priču, a administrator može vidjeti tko je sve ocijenio priču.
		
		Na početnoj stranici aplikacije nalaze se mini prikazi svih već objavljenih priča (eng. thumbnails). Administrator stranice ima ponuđenu opciju stvaranja priče, odabirom koje joj se nudi sučelje za sastavljanje priče. Ako se unutar priče radi o maketi koju administrator želi prodati, nudi mu se opcija da maketu stavi na prodaju. Odabirom te opcije, sučelje se proširuje poljima za unos specifikacija i cijene makete iz priče. Takve makete nisu dostupne u web trgovini, već se gumb za njihovo dodavanje u košaricu nalazi na dnu priče. Budući da se radi o jedinstvenim maketama, kad je maketa prvi put kupljena, gumb za prodaju se onemogućuje i zasivi. Registrirani korisnik na glavnoj stranici ima ponuđenu opciju predlaganja priče. Prilikom predlaganja priče, korisnik bira želi li samo postaviti zahtjev administratoru stranice koju temu da obradi u sljedećoj priči ili želi poslati vlastitu priču. Ukoliko odabere opciju za slanje vlastite priče, pojavljuje mu se isto sučelje kao kad administrator sastavlja vlastitu priču. Administrator ima pretinac u koji mu stižu prijedlozi tema i gotove priče registriranih korisnika i tamo mu je ponuđena opcija odobriti ili odbiti pojedine priče te odgovoriti na predložene teme. Klikom na pojedinu priču, može se pogledati puni tekst priče i/ili galerija slika i videa te se prikazuje prostor za komentare. Komentari registriranih korisnika prikazuju korisničko ime i profilnu sliku, dok su komentari neregistriranih korisnika objavljeni pod rezerviranim korisničkim imenom “Anon” i predefiniranom profilnom slikom.
		
		Korisnici imaju opciju registrirati se. Pritom su njihovi podaci zapamćeni u bazi podataka i koriste se svaki put kad korisnik želi kupovati ili komentirati. Na taj način korisnik ne mora svaki put ispočetka upisivati svoje podatke ako je već registriran. Prilikom registracije (ili kasnije), korisnik ima opciju postavljanja profilne slike. Ako odabere ostati bez profilne slike, automatski mu je pridijeljena predefinirana koja je ista kao za neprijavljenje korisnike. Korisnik također ima svoju stranicu profila koja je vidljiva svim drugim korisnicima. Početne su postavke namještene tako da je od svih podataka samo korisničko ime javno, no korisnik ima opciju na svojoj stranici profila mijenjati postavke privatnosti - može omogućiti vidljivost i drugih podataka (ne podataka o plaćanju poput broja kreditne kartice). Korisnik na svojoj stranici profila također može uređivati sam sadržaj svih svojih podataka (npr. promjena adrese ako se preselio). Korisniku je ponuđena i opcija brisanja vlastitog profila. Ukoliko korisnik odluči izbrisati profil, njegovi se podaci brišu iz sustava i svi komentari koje je napisao prikazuju se kao komentari neprijavljenog korisnika.
		
		Sastavni je dio aplikacije i web trgovina. U web trgovini izložene su standardne makete koje se prodaju. Na početnoj stranici web trgovine, nalaze se samo male sličice koje prikazuju makete (eng. thumbnails). Korisnik klikom na pojedini thumbnail može pregledati opis, dok se administratoru klikom nudi opcija uređivanja ili uklanjanja makete iz web trgovine. Svaka maketa unutar svog opisa sadrži specifikaciju (dimenzije, materijale i boje). Kupcu u web trgovini za svaku se maketu nudi odabir različitih materijala. Taj odabir utječe na cijenu makete te je cijena prikazana tek kada korisnik odabere materijal. Korisnici također imaju opciju ocjenjivanja maketa u webshopu na skali od 1-5 zvjezdica. Administrator ima opciju unošenja standardnih maketa u web trgovinu (u početku je prazna). Pritom mu se za to otvara sučelje unutar kojeg može staviti slike makete, specifikacije te definirati cijenu makete za svaki od dostupnih materijala izrade.
		
		Jedna od pogodnosti koje se nude registriranim korisnicima naručivanje je maketa prema vlastitim idejama i skicama. Na glavnoj stranici postoji gumbić koji vodi korisnika na formular. Korisnik ga popunjava i pritom prilaže svoje skice, slike i tekstualne opise uz navođenje dimenzija i željenih materijala potrebnih za izradu maketa. Tako popunjeni formular šalje se administratoru. Administrator pregledava zahtjev i u slučaju prihvaćanja zahtjeva, kreira cijenu za njega koju šalje natrag korisniku. Potom korisnik ima mogućnost odbijanja ili prihvaćanja te cijene. U slučaju prihvaćanja narudžbe, izrađena će maketa biti poslana na mjesto prebivališta korisnika koje je vezano uz njegov profil.
		
		Administratoru je dostupan popis svih registriranih korisnika. Ukoliko se neki od korisnika počne neprimjereno ponašati, administrator ima opciju zabrane pristupa pojedinom korisniku na stranici profila. Administrator može privremeno zabraniti pristup na određeno vrijeme ili do ručne ponovne dozvole pristupa. 
		
		Već spomenuta funkcionalnost web trgovine zahtijeva i mogućnost plaćanja putem interneta. Nakon odabira željenih proizvoda, pregleda i potvrde sadržaja košarice, korisnik ispunjava formular za plaćanje. Neregistrirani korisnici i registrirani korisnici koji po prvi puta obavljaju online kupnju na ovoj stranici, unose sve svoje osobne podatke u formular. Registrirani korisnici kojima to nije prva kupnja imaju već ispunjeni formular. Sustav pamti povijest svih transakcija te su one dostupne isključivo administratoru.
		
		
		\section{Primjeri u \LaTeX u}
		
		\textit{Ovo potpoglavlje izbrisati.}\\

		U nastavku se nalaze različiti primjeri kako koristiti osnovne funkcionalnosti \LaTeX a koje su potrebne za izradu dokumentacije. Za dodatnu pomoć obratiti se asistentu na projektu ili potražiti upute na sljedećim web sjedištima:
		\begin{itemize}
			\item Upute za izradu diplomskog rada u \LaTeX u - \url{https://www.fer.unizg.hr/_download/repository/LaTeX-upute.pdf}
			\item \LaTeX\ projekt - \url{https://www.latex-project.org/help/}
			\item StackExchange za Tex - \url{https://tex.stackexchange.com/}\\
		
		\end{itemize} 	


		
		\noindent \underbar{podcrtani tekst}, \textbf{podebljani tekst}, 	\textit{nagnuti tekst}\\
		\noindent \normalsize primjer \large primjer \Large primjer \LARGE {primjer} \huge {primjer} \Huge primjer \normalsize
				
		\begin{packed_item}
			
			\item  primjer
			\item  primjer
			\item  primjer
			\item[] \begin{packed_enum}
				\item primjer
				\item[] \begin{packed_enum}
					\item[1.a] primjer
					\item[b] primjer
				\end{packed_enum}
				\item primjer
			\end{packed_enum}
			
		\end{packed_item}
		
		\noindent primjer url-a: \url{https://www.fer.unizg.hr/predmet/proinz/projekt}
		
		\noindent posebni znakovi: \# \$ \% \& \{ \} \_ 
		$|$ $<$ $>$ 
		\^{} 
		\~{} 
		$\backslash$ 
		
		\begin{longtabu} to \textwidth {|X[8, l]|X[8, l]|X[16, l]|} %definicija širine tablice, širine stupaca i poravnanje
			
			%definicija naslova tablice
			\hline \multicolumn{3}{|c|}{\textbf{naslov unutar tablice}}	 \\[3pt] \hline
			\endfirsthead
			
			%definicija naslova tablice prilikom prijeloma
			\hline \multicolumn{3}{|c|}{\textbf{naslov unutar tablice}}	 \\[3pt] \hline
			\endhead
			
			\hline 
			\endlastfoot
			
			\rowcolor{LightGreen}IDKorisnik & INT	&  	Lorem ipsum dolor sit amet, consectetur adipiscing elit, sed do eiusmod  	\\ \hline
			korisnickoIme	& VARCHAR &   	\\ \hline 
			email & VARCHAR &   \\ \hline 
			ime & VARCHAR	&  		\\ \hline 
			\cellcolor{LightBlue} primjer	& VARCHAR &   	\\ \hline 
			
		\end{longtabu}
		

		\begin{table}[H]
			
			\begin{longtabu} to \textwidth {|X[8, l]|X[8, l]|X[16, l]|} 
				
				\hline 
				\endfirsthead
				
				\hline 
				\endhead
				
				\hline 
				\endlastfoot
				
				\rowcolor{LightGreen}IDKorisnik & INT	&  	Lorem ipsum dolor sit amet, consectetur adipiscing elit, sed do eiusmod  	\\ \hline
				korisnickoIme	& VARCHAR &   	\\ \hline 
				email & VARCHAR &   \\ \hline 
				ime & VARCHAR	&  		\\ \hline 
				\cellcolor{LightBlue} primjer	& VARCHAR &   	\\ \hline 
				
				
			\end{longtabu}
	
			\caption{\label{tab:referencatablica} Naslov ispod tablice.}
		\end{table}
		
		
		%unos slike
		\begin{figure}[H]
			\includegraphics[scale=0.4]{slike/aktivnost.PNG} %veličina slike u odnosu na originalnu datoteku i pozicija slike
			\centering
			\caption{Primjer slike s potpisom}
			\label{fig:promjene}
		\end{figure}
		
		\begin{figure}[H]
			\includegraphics[width=.9\linewidth]{slike/aktivnost.PNG} %veličina u odnosu na širinu linije
			\caption{Primjer slike s potpisom 2}
			\label{fig:promjene2} %label mora biti drugaciji za svaku sliku
		\end{figure}
		
		Referenciranje slike \ref{fig:promjene2} u tekstu.
		
		\eject
		
	