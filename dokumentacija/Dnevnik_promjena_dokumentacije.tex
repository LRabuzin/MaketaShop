\chapter{Dnevnik promjena dokumentacije}
		
		\textbf{\textit{Kontinuirano osvježavanje}}\\
				
		
		\begin{longtabu} to \textwidth {|X[2, l]|X[13, l]|X[3, l]|X[3, l]|}
			\hline \multicolumn{1}{|l|}{\textbf{Rev.}}	& \multicolumn{1}{l|}{\textbf{Opis promjene/dodatka}} & \multicolumn{1}{|l|}{\textbf{Autori}} & \multicolumn{1}{l|}{\textbf{Datum}} \\[3pt] \hline
			\endfirsthead
			
			\hline \multicolumn{1}{|l|}{\textbf{Rev.}}	& \multicolumn{1}{l|}{\textbf{Opis promjene/dodatka}} & \multicolumn{1}{|l|}{\textbf{Autori}} & \multicolumn{1}{l|}{\textbf{Datum}} \\[3pt] \hline
			\endhead
			
			\hline 
			\endlastfoot
			
			0.1 & Dopunjen predložak	& Rabuzin & 11.10.2020. 		\\[3pt]
			\hline
			0.2 & Opis projektnog zadatka	& Rabuzin i Šarić & 15.10.2020. 		\\[3pt]
			\hline 
			0.2.1 & Detaljnije razrađen opis projektnog zadatka	& Rabuzin i Šarić & 17.10.2020. 		\\[3pt]
			\hline
			0.3 & Dodani obrasci uporabe u latex	& Sičić & 21.10.2020. 		\\[3pt]
			\hline
			0.3.1 & Uređeni obrasci uporabe i napravljeni UML dijagrami	& Zmiša i Hrestak & 22.10.2020. 		\\[3pt]
			\hline
			%U komentarima je ostavljeno kako je izgledao predložak radi reference za budućnost
			%0.2	& Dopisane upute za povijest dokumentacije.\newline Dodane reference. & Jović & 24.08.2013. 	\\[3pt] \hline 
			%0.5 & Dodan \textit{Use Case} dijagram i jedan sekvencijski dijagram, funkcionalni i nefunkcionalni zahtjevi i dodatak A & Ivošević & 25.08.2013. \\[3pt] \hline 
			%0.6 & Arhitektura i dizajn sustava, algoritmi i strukture podataka & Grudenić & 26.08.2013. \\[3pt] \hline 
			%0.8 & Povijest rada i trenutni status implementacije,\newline Zaključci i plan daljnjeg rada & Ivošević & 28.08.2013. \\[3pt] \hline 
			%0.9 & Opisi obrazaca uporabe & Jović & 07.09.2013. \\[3pt] \hline 
			%0.10 & Preveden uvod & Jović & 08.09.2013. \\[3pt] \hline 
			%0.11 & Sekvencijski dijagrami & Žužak & 09.09.2013. \\[3pt] \hline 
			%0.12.1 & Započeo dijagrame razreda & Horvat & 10.09.2013. \\[3pt] \hline 
			%0.12.2 & Nastavak dijagrama razreda & Horvat & 11.09.2013. \\[3pt] \hline 
			%\textbf{1.0} & Verzija samo s bitnim dijelovima za 1. ciklus & Ivošević & 11.09.2013. \\[3pt] \hline 
			%1.1 & Uređivanje teksta -- funkcionalni i nefunkcionalni zahtjevi & Grudenić \newline Jović & 14.09.2013. \\[3pt] \hline 
			%1.2 & Manje izmjene:Timer - Brojilo vremena & Grudenić & 15.09.2013. \\[3pt] \hline 
			%1.3 & Popravljeni dijagrami obrazaca uporabe & Jović & 15.09.2013. \\[3pt] \hline 
			%1.5 & Generalna revizija strukture dokumenta & Ivošević & 19.09.2013. \\[3pt] \hline 
			%1.5.1 & Manja revizija (dijagram razmještaja) & Jović & 20.09.2013. \\[3pt] \hline 
			%\textbf{2.0} & Konačni tekst predloška dokumentacije  & Ivošević & 28.09.2013. \\[3pt] \hline 
			&  &  & \\[3pt] \hline
			
			
		\end{longtabu}
	
	
		\textit{Moraju postojati glavne revizije dokumenata 1.0 i 2.0 na kraju prvog i drugog ciklusa. Između tih revizija mogu postojati manje revizije već prema tome kako se dokument bude nadopunjavao. Očekuje se da nakon svake značajnije promjene (dodatka, izmjene, uklanjanja dijelova teksta i popratnih grafičkih sadržaja) dokumenta se to zabilježi kao revizija. Npr., revizije unutar prvog ciklusa će imati oznake 0.1, 0.2, …, 0.9, 0.10, 0.11.. sve do konačne revizije prvog ciklusa 1.0. U drugom ciklusu se nastavlja s revizijama 1.1, 1.2, itd.}